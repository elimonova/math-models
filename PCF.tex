\section{PCF}
\renewcommand{\ll}[1]{\lambda^{l}_{#1}}
\newcommand{\rl}[1]{r^{l}_{#1}}
\newcommand{\lb}[1]{\lambda^{b}_{#1}}
\newcommand{\rb}[1]{r^{b}_{#1}}
\newcommand{\tb}[1]{\overline{t_{#1 b}}}
\newcommand{\tl}[1]{\overline{t_{#1 l}}}
\newcommand{\Tdb}[1]{T_{DATA}^{#1,b}}
\newcommand{\Tdl}[1]{T_{DATA}^{#1,l}}

\subsection{PCF-0}
Считаем, что CP много меньше CFP.
Кадры опроса CF\_POLL{+DATA{+ACK}}

Базовая станция (БС) работает в насыщении и передает пакеты длины $r_b$, которые доходят успешно с вероятностью $1-p_b$.

\begin{tabular}{l l l}
$p_{b}$ 	&---	&вероятность неуспешной передачи кадра с данными <<вниз>>  (downlink) \\
$p_{c}$		&---	&вероятность неуспешной передачи служебного кадра <<вниз>> (downlink)   \\
\end{tabular}

Также имеется $N$ станций с насыщении $(r_l, p_l)$:

\begin{tabular}{l l l}
$r_{l}$ 	&---	&длина кадров с данными \\
$p_{l}$		&---	&вероятность неуспешной передачи служебного кадра <<вверх>> (uplink)   \\
\end{tabular}

Найти пропускные способности $S_{down}$ и $S_{up}$.
\begin{align*}
S_{down} &= \frac{\alpha_b r_b}{T_{slot}},  & \alpha_b &= (1-p_b)(1-p_l); \\
S_{up} &= \frac{\alpha_l r_l}{T_{slot}},  & \alpha_l &= (1-p_l)(1-p_b)^2,
\end{align*}
где 
\begin{equation}
\notag
T_{slot} = T_{DATA}^b + p_b \cdot PIFS + (1-p_b)(SIFS + T_{DATA}^l + SIFS)
\end{equation}
Из выражений для $\alpha_l$ и $\alpha_b$ видно, что $\alpha_l < \alpha_b$, т.~е. восходящий (uplink) поток находится в менее выгодном положении, чем нисходящий (downlink).

Может лучше посылать DATA и служебные кадры раздельно?

\subsection{PCF-1}
Шлем все кадры отдельно, тогда 
\begin{align*}
\alpha_b &= (1-p_b)(1-p_c)\\
\alpha_l &= (1-p_l)(1-p_c)^2
\end{align*}

\begin{equation}
\notag
T_{slot} = T_{DATA} + p_b PIFS + (1-p_b)(2SIFS + ACK) + CF\_POLL + p_c PIFS + (1-p_c)(T_{DATA}^l + 2SIFS + (1-p_l)ACK)
\end{equation}

\subsection{PCF-2}
Имеется $N$ станций подключенные к базовой станции (БС). На БС $N$ очередей: по одной на очереди на каждую станцию. 

В каждую очередь $i$ базовой станции поступает пуассоновский поток интенсивности $\lb{i}$ для передачи соответствующей станции $i$ (downlink). В свою очередь на каждой станции $i$ генерируется пуассоновский поток пакетов интенсивности $\ll{i}$ для передачи на БС (uplink). 

\begin{tabular}{l l l}
$\tb{i}$ 	&---	&среднее время обслуживания пакета <<вниз>>  (downlink) \\
$\tl{i}$	&---	&среднее время обслуживания пакета <<вверх>> (uplink)   \\
$t_0$		&---	&служебный кадр (CF\_POLL) \\
\end{tabular}

Если $\theta_s$ --- среднее время обслуживания станции номер $s$, то 
$T_{cycle} = \sum\limits_{s = 1}^{N} \theta_s$ --- среднее время обслуживания всех станций.
\begin{equation}
\notag
\theta_s = \rho_{sb} \Tdb{s} + (1 - \rho_{sb})t_0 + \beta_{sb} PIFS + (1-\beta_{sb})(\rho_{sl}\Tdl{s} + (1- \rho_{sl})t_{null} + SIFS),
\end{equation}
где  

\begin{tabular}{l l l}
$\rho_{sb}$ 	&---	&вероятность того, что очередь на БС для станции $s$ не пуста \\
$\rho_{sl}$		&---	&вероятность того, что очередь на станции $s$ не пуста  \\
$\beta_{sb}$	&---	&вероятность неудачного опроса:\\
\end{tabular}
\begin{equation}
\beta_{sb} = \rho_{sb} p^b_s + (1 - \rho_{sb})p_c.
\end{equation}

Теперь можем найти $\tb{i}$:
\begin{equation}
\notag
\tb{i} = \frac{1-\alpha_{ib}}{\alpha_{ib}} T_{cycle} + \rho_{ib} T_{cycle} + (1 - \rho_{ib}) (T_{cycle}-t_{ib}').
\end{equation}
$t_{ib}'$ --- среднее время до начала обслуживания при поступлении в пустую очередь.
\begin{equation}
t_{ib}' = \frac{\int\limits_{0}^{T_{cycle}} x\lb{i}e^{-\lb{i}x}dx}{1-e^{-\lb{i}T_{cycle}}} = \frac{1-(1+\lb{i}T_{cycle})e^{-\lb{i}T_{cycle}}}{\lb{i}(1-e^{-\lb{i}T_{cycle}})}
\end{equation}

\begin{eqnarray}
\alpha_{ib} & = & (1 - p_{i}^{b})((1-\rho_{il})(1-p_c)+\rho_{il}(1-p_{i}^l)) \\
\alpha_{il} & = & (1 - \beta_{ib})(1-p_i^l)(1-\beta_{{(i+1)\mod N},b}) \\
\end{eqnarray}

